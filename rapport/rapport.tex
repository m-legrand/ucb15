\documentclass[12pt,a4paper]{report}

\usepackage[utf8]{inputenc}
\usepackage[T1]{fontenc}
\usepackage[francais,frenchb]{babel}
\usepackage[top=1.5cm, bottom=1.5cm, left=2cm, right=2cm]{geometry}
\usepackage{lmodern}
\usepackage{amsmath}
\usepackage{amssymb}
\usepackage{graphicx}
\usepackage{mathrsfs}
\usepackage{amsthm}
\usepackage{verbatim}
\usepackage{moreverb}
\usepackage{textcomp}

\usepackage[affil-it]{authblk}

\newcommand{\HRule}{\rule{\linewidth}{0.5mm}}

\begin{document}

\newcounter{Thm}
\newtheorem{prop}[Thm]{Proposition}
\newtheorem{lemme}[Thm]{Lemme}
\newtheorem{cor}[Thm]{Corollaire}
\newtheorem{defi}{Définition}
\newtheorem{theo}[Thm]{Théorème}

\newtheorem*{demonstration}{Démonstration}
\newenvironment{demo}{\begin{samepage} \begin{demonstration} \upshape }{ \end{demonstration} \end{samepage}}
\newtheorem*{remarque}{Remarque}
\newenvironment{rmq}{\begin{samepage} \begin{remarque} \upshape }{ \end{remarque} \end{samepage}}
\newtheorem*{exercice}{Exercice d'application}
\newenvironment{exo}{\begin{samepage} \begin{exercice} \upshape }{ \end{exercice} \end{samepage}}
\newtheorem*{exemple}{Exemple}
\newenvironment{ex}{\begin{samepage} \begin{exemple} \upshape }{ \end{exemple} \end{samepage}}
\newtheorem*{remarques}{Remarques}
\newenvironment{rmqs}{\begin{samepage} \begin{remarques} \upshape }{ \end{remarques} \end{samepage}}

\newcommand{\rA}{\mathcal{A}}
\newcommand{\rC}{\mathcal{C}}
\newcommand{\rD}{\mathcal{D}}
\newcommand{\rE}{\mathcal{E}}
\newcommand{\rF}{\mathcal{F}}
\newcommand{\rG}{\mathcal{G}}
\newcommand{\rP}{\mathcal{P}}
\newcommand{\rH}{\mathcal{H}}
\newcommand{\rR}{\mathcal{R}}
\newcommand{\rL}{\mathcal{L}}
\newcommand{\rM}{\mathcal{M}}
\newcommand{\bD}{\mathbb{D}}
\newcommand{\bK}{\mathbb{K}}
\newcommand{\bQ}{\mathbb{Q}}
\newcommand{\bR}{\mathbb{R}}
\newcommand{\bU}{\mathbb{U}}
\newcommand{\bZ}{\mathbb{Z}}
\newcommand{\bN}{\mathbb{N}}
\newcommand{\bC}{\mathbb{C}}
\newcommand{\bP}{\mathbb{P}}
\newcommand{\bF}{\mathbb{F}}
\newcommand{\eps}{\varepsilon}

\renewcommand{\thechapter}{\Roman{chapter}}

\ifpdf
\DeclareGraphicsExtensions{.pdf, .jpg, .tif}
\else
\DeclareGraphicsExtensions{.eps, .jpg}
\fi

%%%%%%%%%%%%%%%%%%%%%%%%%%%%%%%%%%%%%%%%%%%%%%%%%%%%%%%%

\begin{titlepage}
\begin{center}

~\\[1.5cm]
\textsc{\Large Rapport de stage de M1}\\[0.5cm]

% Title
\HRule \\[0.4cm]
{ \huge \bfseries Markov Cluster Algorithm\\[0.4cm] }

\HRule \\[1.5cm]

{\large Maxime \textsc{Legrand}}\\
~\\
{\large \emph{École Normale Supérieure, Paris}}\\
~\\[1cm]
{\large Encadré par :}\\
~\\
{\large Noureddine El Karoui}\\
~\\
{\large \emph{University of California, Berkeley}}\\

\vfill

\includegraphics[height=0.3\textwidth]{images/logoens.png} \hspace{2.5cm}
\includegraphics[height=0.3\textwidth]{images/logoucb.jpg}~\\

\vfill

% Bottom of the page
{\large 28 août 2015}
\end{center}
\end{titlepage}

\tableofcontents

\newpage

\chapter*{Introduction}
\addcontentsline{toc}{chapter}{Introduction}

\stepcounter{chapter}
\chapter*{\thechapter . Autour de MCL ordinaire}
\addcontentsline{toc}{chapter}{\thechapter . Autour de MCL ordinaire}

\section{MCL classique : spécification et sémantique}

% Détail de l'algorithme
% Interprétation intuitive
% Implantation basique

\section{Représentation des réseaux et évaluation visuelle}

% Présentation de netplot
% Étude de cas généraux, et de cas défavorables par évaluation visuelle

\section{Inférence d'un coefficient de doute}

% Suite logique de la section précédente en introduisant des valeurs associées
% Repérage et traitement semi-automatique des conditions défavorables

\stepcounter{chapter}
\chapter*{\thechapter . Variantes de MCL}
\addcontentsline{toc}{chapter}{\thechapter . Variantes de MCL}

\section{Algorithme $r$-adaptatif}

\section{Algorithme à lien forcé}

\section{Algorithme à séparation forcée}

\stepcounter{chapter}
\chapter*{\thechapter . Propriétés}
\addcontentsline{toc}{chapter}{\thechapter . Propriétés}

\section{Analyse par perturbations}

\section{Application sur des données réelles}

\section{Comparaison avec des algorithmes classiques de classification}

\subsection*{Spectral clustering}

\subsection*{Hierarchical clustering}

\subsection*{Minimum-cut clustering}

\subsection*{Girvan-Newman clustering}

\chapter*{Conclusion}
\addcontentsline{toc}{chapter}{Conclusion}

\nocite{vandongen00}
\bibliography{rapport}
\bibliographystyle{plain}

\end{document}